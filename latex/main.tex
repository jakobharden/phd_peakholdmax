% !BIB TS-program = biblatex
% !TeX spellcheck = en_BR
%
%#######################################################################################################################
% LICENSE
%
% "main.tex" (C) 2025 by Jakob Harden (Graz University of Technology) is licensed under a Creative Commons Attribution 4.0 International license.
%
% License deed: https://creativecommons.org/licenses/by/4.0/
% Author email: jakob.harden@tugraz.at, jakob.harden@student.tugraz.at, office@jakobharden.at
% Author website: https://jakobharden.at/wordpress/
% Author ORCID: https://orcid.org/0000-0002-5752-1785
%
% This file is part of the PhD thesis of Jakob Harden.
%#######################################################################################################################
%
% Beamer documentation: https://www.beamer.plus/Structuring-Presentation-The-Local-Structure.html
%
% preamble
\documentclass[11pt,aspectratio=169]{beamer}
\usepackage[utf8]{inputenc}
\usepackage[LGR,T1]{fontenc}
\usepackage[ngerman,english]{babel}
\usepackage{hyphenat}
\usepackage{lmodern}
\usepackage{blindtext}
\usepackage{multicol}
\usepackage{graphicx}
\usepackage{tikz}
\usetikzlibrary{calc,fpu,plotmarks}
\usepackage{pgfplots}
\pgfplotsset{compat=1.17}
\usepgflibrary{fpu}
\usepackage{amsmath}
\usepackage{algorithm}
\usepackage{algpseudocode}
\usepackage{hyperref}
\usepackage[backend=biber,style=numeric]{biblatex}
\addbibresource{biblio.bib}

% breakable algorithm
% by https://github.com/Tedxz/xjtuthesis-x/issues/1
\makeatletter
\newenvironment{breakablealgorithm}
{% \begin{breakablealgorithm}
	\begin{center}
		\refstepcounter{algorithm}% New algorithm
		\hrule height.8pt depth0pt \kern2pt% \@fs@pre for \@fs@ruled
		\renewcommand{\caption}[2][\relax]{% Make a new \caption
			{\raggedright\textbf{\ALG@name~\thealgorithm} ##2\par}%
			\ifx\relax##1\relax % #1 is \relax
			\addcontentsline{loa}{algorithm}{\protect\numberline{\thealgorithm}##2}%
			\else % #1 is not \relax
			\addcontentsline{loa}{algorithm}{\protect\numberline{\thealgorithm}##1}%
			\fi
			\kern2pt\hrule\kern2pt
		}
	}{% \end{breakablealgorithm}
		\kern2pt\hrule\relax% \@fs@post for \@fs@ruled
	\end{center}
}
\makeatother

%
% text blocks
\def\PresTitle{Local extreme value detection for sinusoidal signals corrupted by noise}
\def\PresSubTitle{A numerical study on (damped) sinusoidal signals}
\def\PresDate{25${}^{\text{st}}$ of May, 2025}
\def\PresFootInfo{My PhD thesis, research in progress ...}
\def\PresAuthorFirstname{Jakob}
\def\PresAuthorLastname{Harden}
\def\PresAuthor{\PresAuthorFirstname{} \PresAuthorLastname{}}
\def\PresAuthorAffiliation{Graz University of Technology}
\def\PresAuthorAffiliationLocation{\PresAuthorAffiliation{} (Graz, Austria)}
\def\PresAuhtorWebsite{jakobharden.at}
\def\PresAuhtorWebsiteURL{https://jakobharden.at/wordpress/}
\def\PresAuhtorEmailFirst{jakob.harden@tugraz.at}
\def\PresAuhtorEmailSecond{jakob.harden@student.tugraz.at}
\def\PresAuhtorEmailThird{office@jakobharden.at}
\def\PresAuthorOrcid{0000-0002-5752-1785}
\def\PresAuthorOrcidURL{https://orcid.org/0000-0002-5752-1785}
\def\PresAuthorLinkedin{jakobharden}
\def\PresAuthorLinkedinURL{https://www.linkedin.com/in/jakobharden/}
\def\PresCopyrightType{ccby} % one of: copyright, ccby, ccysa
%
% Beamer theme adaptations
%   type:        Presentation
%   series:      Research in progress (RIP)
%   description: This theme is designed to present preliminary research results.
\input{adaptthemePresRIP}
%
% Load Octave to TeX tool
% TeX commands to conveniently use serialized dataset content
%
%#######################################################################################################################
% LICENSE
%
% "oct2texdefs.tex" (C) 2024 by Jakob Harden (Graz University of Technology) is licensed under a Creative Commons Attribution 4.0 International license.
%
% License deed: https://creativecommons.org/licenses/by/4.0/
% Author email: jakob.harden@tugraz.at, jakob.harden@student.tugraz.at, office@jakobharden.at
% Author website: https://jakobharden.at/wordpress/
% Author ORCID: https://orcid.org/0000-0002-5752-1785
%
% This file is part of the PhD thesis of Jakob Harden.
%#######################################################################################################################
%
%
%-------------------------------------------------------------------------------
% Load etoolbox and other required pgf packages
\usepackage{etoolbox} % if clauses
\usepackage{pgf, pgfmath, pgfplots, pgfplotstable} % pgf functions
%
%-------------------------------------------------------------------------------
% Structure path prefix
% Note: The prefix is used to abbreviate long structure paths (variable names)
%
% Define default value of structure path prefix
% Do not change that value unless you know what you are doing!
\gdef\OTpfx{oct2tex}
%
% Set structure path prefix to a user defined value
%   Parameter #1: user defined prefix (string without whitespace)
%   Usage: \OTsetpfx{oct2tex.my.pre.fix}
\newcommand{\OTsetpfx}[1]{\ifstrempty{#1}{\gdef\OTpfx{oct2tex}}{\gdef\OTpfx{#1}}}
%
% Reset structure path prefix to default value
\newcommand{\OTresetpfx}{\gdef\OTpfx{oct2tex}}
%
%-------------------------------------------------------------------------------
% Use serialized content from data structures in the document
%
% Use structure variable
%   Parameter #1: variable name (structure path)
%   Usage: \OTuse{my.struct.path.to.content.value}
\newcommand{\OTuse}[1]{\csname \OTpfx.#1\endcsname}
%
% Use structure variable, fixed digit floating point number
%   Parameter #1: variable name (structure path)
%   Parameter #2: number of digits to display
%   Usage: \OTusefixed{my.struct.path.to.content.value}{2}
\newcommand{\OTusefixed}[2]{%
	\pgfkeys{%
		/pgf/number format/.cd,%
		fixed,%
		precision=#2,%
		1000 sep={.}%
	}%
	\pgfmathprintnumber{\OTuse{#1}}%
}
%
% Read tabulated value from structure and store result in the LaTeX command \OTtab
%   Parameter #1: variable name (structure path)
%   Usage: \OTread{my.struct.path.to.table}
\newcommand{\OTread}[1]{\pgfplotstableread[col sep=semicolon,trim cells]{\OTpfx.#1}\OTtab}
%
% Read CSV file and store result in the LaTeX command \OTtabcsv
%   Parameter #1: CSV file name
%   Usage: \OTreadcsv{csv_filename}
\newcommand{\OTreadcsv}[1]{\pgfplotstableread[col sep=semicolon,trim cells]{#1}\OTtabcsv}

%
% graphics path
\graphicspath{{../octave/results/test_peakholdmax}}
%
% computation result path
\newcommand{\RPATH}{../octave/results/test_peakholdmax}
%
%#######################################################################################################################
\begin{document}
	% set title page items
	\author{\PresAuthor{} (\PresAuthorAffiliation{})}
	\title{\PresTitle{}}
	\subtitle{\PresSubTitle{}}
	%\logo{}
	%\institute{}
	\date{\PresDate{}}
	%\subject{}
	%\setbeamercovered{transparent}
	%\setbeamertemplate{navigation symbols}{}
	%
	%-------------------------------------------------------------------------------------------------------------------
	\begin{frame}[plain]
		\maketitle
	\end{frame}
	%
	%-------------------------------------------------------------------------------------------------------------------
	\section*{Abstract}
	\begin{frame}
		\frametitle{Abstract}
		% german text
%		Die Analyse von Ultraschallsignalen ist von der Analyse der Signaldaten im Zeitbereich geprägt. Beim Ultraschall-Puls Transmissionsverfahren zielt die Signalanalyse zumeist auf das Detektieren des Ansatzpunktes der eintreffenden Kompressions- oder Scherwelle ab. Damit lässt sich die Schalllaufzeit, und bei bekannter Messdistanz auch die Schallgeschwindigkeit als wesentlichen Materialparameter bestimmen.
%		
%		Dieser Ansatzpunkt ist aufgrund des Signalrauschens und eventuell vorhandener Interferenzen nicht immer einfach zu lokalisieren. Als Hilfsmittel und zur Stabilisierung der Signalanalyse ist es sinnvoll, das erste lokale Extremum (Minimum oder Maximum) direkt hinter dem Ansatzpunkt der Kompressions- oder Scherwelle zu bestimmen. Damit kann das Suchintervall für den Ansatzpunkt deutlich eingeschränkt und die Robustheit der gesamten nachfolgenden Analyse erhöht werden.
%		
%		Wie die Erfahrung zeigt, ist das erste lokale Minimum oder Maximum nicht immer mit dem globalen Minimum oder Maximum ident. Dieser Umstand führt sehr leicht zu einer Falsch-Detektion. In dieser Arbeit wird ein Suchverfahren vorgestellt, das die robuste Detektion des ersten lokalen Extremums ermöglicht und gleichzeitig einen moderaten Berechnungsaufwand mit sich bringt.
		\begin{abstract}
			The analysis of ultrasonic signals is characterised by the analysis of signal data in the time domain. In the ultrasonic pulse transmission method, the signal analysis is often aimed at detecting the onset point of the incoming compression or shear wave. This makes it possible to determine the sound propagation time and, if the measuring distance is known, the speed of sound as an essential material parameter.
			This starting point is not always easy to localise due to signal noise and possible interference. As an aid and to stabilise the signal analysis, it is useful to estimate the location of the first local extreme value (minimum or maximum) directly behind the onset point of the compression or shear wave. This can significantly reduce the search interval for the onset point and increase the robustness of the entire subsequent analysis.
			Experience shows that the first local minimum or maximum is not always identical to the global minimum or maximum. This circumstance very easily leads to a false detection. In this work, a search method is presented that enables the robust detection of the first local extreme and comes with a moderate computational effort.
		\end{abstract}
	\end{frame}
	%
	%-------------------------------------------------------------------------------------------------------------------
	\section{Introduction}
	\begin{frame}
		\frametitle{Introduction}
		\begin{itemize}
			\item \textcolor{RIPtitlecol}{WHAT}
			\begin{itemize}
				\item Ultrasound signals, compression/primary waves (P-wave), shear/secondary waves (S-wave)
				\item Locate the first local extreme value (minimum or maximum)
			\end{itemize}
			\item \textcolor{RIPtitlecol}{WHY}
			\begin{itemize}
				\item Narrow down the search interval for the P-wave or S-wave onset point detection
				\item Increase the robustness of subsequent signal analysis
			\end{itemize}
			\item \textcolor{RIPtitlecol}{HOW}
			\begin{itemize}
				\item Loop over the signal index until a predefined number of following signal amplitudes are \textbf{lower} (maximum) or \textbf{higher} (minimum) than the current amplitude.
			\end{itemize}
			\item \textcolor{RIPtitlecol}{USAGE}
				\begin{itemize}
					\item P-wave and S-wave signal responses (natural signals)
					\item Normalise signal amplitudes w.r.t. the first local maximum
				\end{itemize}
			\item \textcolor{RIPtitlecol}{HIGHLIGHT} \textbf{robust method, moderate computational effort}
		\end{itemize}
	\end{frame}
	%
	%-------------------------------------------------------------------------------------------------------------------
	\section{Materials \& Methods}
	\begin{frame}
		\frametitle{Materials \& Methods I}
		\begin{itemize}
			\setlength\itemsep{0.5em}
			\item \textcolor{RIPtitlecol}{Materials}
			\begin{itemize}
				\setlength\itemsep{0.5em}
				\item (Damped) sinusoidal signals corrupted by additive noise
				\item discrete-time signals typical for ultrasonic pulse transmission tests
			\end{itemize}
			\item \textcolor{RIPtitlecol}{Methods}
			\begin{itemize}
				\setlength\itemsep{0.5em}
				\item Maximum detection algorithm (see page \pageref{algo:detection})
				\item Numerical study 1: evaluate the impact of noise (see page \pageref{algo:detection})
			\end{itemize}
		\end{itemize}
	\end{frame}
	%
	\begin{frame}
		\frametitle{Materials \& Methods II}
		\small
		\begin{tabbing}
			\textbf{Detection algorithm}\label{algo:detection} \\
			\hspace{0.4cm} \= \hspace{0.4cm} \= \hspace{0.4cm} \= \hspace{0.5cm} \= \kill
			 \> Given: $x \ldots$~signal, $n_i \ldots$~detection start index, $c_{\lim} \ldots$~counter limit \\
			 \> Optional: $L_d \ldots$~detection window length \\
			 1: \> $[v_{\max}, n_{\max}]$ = \textsc{PeakHoldMax}($x, n_i, c_{\lim}, L_d$) \\
			 2: \> \> $v_m, n_m \gets \max(x[n_i \leq n \leq |x|])$ \\
			 3: \> \> $L_{d,\max} \gets |x| - n_i + 1$ \\
			 4: \> \> \textbf{if} ($L_d = [\;]$) \\
			 5: \> \> \> $L_d \gets L_{d,\max}$ \\
			 6: \> \> \textbf{else} \\
			 7: \> \> \> $L_d \gets \min(L_d, L_{d,\max})$ \\
			 8: \> \> \textbf{endif} \\
			 9: \> \> $x[n] \gets 0, \; \forall \; x[n] \leq \frac{v_m}{5}$ \\
			 10: \> \> $a \gets min(x[n_i \leq n \leq |x|])$ \\
			 11: \> \> $c \gets 0$ \\
		\end{tabbing}
	\end{frame}
	%
	\begin{frame}
		\frametitle{Materials \& Methods III}
		\small
		\begin{tabbing}
			\hspace{0.5cm} \= \hspace{0.5cm} \= \hspace{0.5cm} \= \hspace{0.5cm} \= \kill
			12: \> \> \textbf{for} $n \in n_i \leq n \leq (n_i + L_d - 1)$ \\
			13: \> \> \> \textbf{if} ($x[n] \geq a$)
\\
			14: \> \> \> \> $a \gets x[n]$ \\
			15: \> \> \> \> $c = 0$ \\
			16: \> \> \> \textbf{else}
\\
			17: \> \> \> \> $c \gets c + 1$
\\
			18: \> \> \> \textbf{endif}
\\
			19: \> \> \> \textbf{if} ($c = c_{\lim}$) \\
			20: \> \> \> \> $v_{\max} \gets a$ \\
			21: \> \> \> \> $n_{\max} \gets n - c_{\lim}$ \\
			22: \> \> \> \> \textbf{break}
\\
			23: \> \> \> \textbf{endif}
\\
			24: \> \> \textbf{endfor} \\
			24: \> \textbf{endfunction} \\
		\end{tabbing}
	\end{frame}
	%
	\begin{frame}
		\frametitle{Materials \& Methods III}
		To study the algorithm, a parameter variation was carried out using the following parameters.
		\begin{itemize}
			\setlength\itemsep{0.5em}
			\item \textcolor{RIPtitlecol}{Numerical study - parameter variation}
			\begin{itemize}
				\setlength\itemsep{0.5em}
				\item Constant: $A = 1 \quad [V]$
				\item Constant: unbiased ACF estimator
				\item Variation 1: $N_1 = (64, 512, 1024, 4096)$ samples
				\item Variation 2: $SNR = (5, 10, 15, 20)$ dB
				\item Variation 3: $DF = (0, 2, 4)$
				\item Variation 4: $N_{cy} = (1,\ldots,5)$ samples, subdivided into 81 steps
				\item Variation 5: $N_{mc} = 500$ turns, Monte-Carlo test
			\end{itemize}
		\end{itemize}
	\end{frame}
	%
	%-------------------------------------------------------------------------------------------------------------------
	\section{Results}
	%
	\begin{frame}
		\frametitle{Results \textendash{} Signal power estimates, example I}
		\begin{columns}[t]
			\begin{RIPcolleft}
				\begin{figure}
					%\includegraphics[width=100mm,trim= 5mm 0mm 5mm 50mm]{xxx}
				\end{figure}
			\end{RIPcolleft}
			\begin{RIPcolright}
				\textbf{xxx}\\
				xxx\\
				\vspace*{.5em}
				\textbf{Observations}\\
				\begin{itemize}
					\item xxx
				\end{itemize}
			\end{RIPcolright}
		\end{columns}
	\end{frame}
	%
	%-------------------------------------------------------------------------------------------------------------------
	\section{Conclusions}
	\begin{frame}
		\frametitle{Conclusions}
		% briefly summarize all observations
		xxx
		\vspace*{1em}
		\begin{itemize}
			\item xxx
		\end{itemize}
	\end{frame}
	%
	%-------------------------------------------------------------------------------------------------------------------
	\section{Outlook}
	\begin{frame}
		\frametitle{Outlook}
		% briefly describe further and connected research
		xxx
		\vspace*{1em}
		\begin{itemize}
			\item xxx
		\end{itemize}
		\vspace*{.5em}
		\small \textbf{Note:} To support the future development of the proposed method, the supplementary GNU Octave code\autocite{progcode} and the \LaTeX{} code\autocite{texcode} of this presentation is made available publicly under open licenses.
	\end{frame}
	%
	%-------------------------------------------------------------------------------------------------------------------
	\section*{References}
	\begin{frame}[noframenumbering]
		\frametitle{References}
		\printbibliography
	\end{frame}
	%
	%===================================================================================================================
	\appendix
	\section{\appendixname}
	%
	\begin{frame}
		\frametitle{\appendixname{} \textendash{} xxx}
		xxx
	\end{frame}
	%
	\begin{frame}[noframenumbering]
		\frametitle{\appendixname{} \textendash{} Author information}
		\RIPauthorinfo{}
	\end{frame}
	\begin{frame}[noframenumbering]
		\frametitle{\appendixname{} \textendash{} Document license}
		\expandafter\RIPcopyrightinfo\expandafter{\PresCopyrightType}
	\end{frame}
\end{document}
