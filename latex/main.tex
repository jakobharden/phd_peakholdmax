% !BIB TS-program = biblatex
% !TeX spellcheck = en_US
%
%#######################################################################################################################
% LICENSE
%
% "main.tex" (C) 2025 by Jakob Harden (Graz University of Technology) is licensed under a Creative Commons Attribution 4.0 International license.
%
% License deed: https://creativecommons.org/licenses/by/4.0/
% Author email: jakob.harden@tugraz.at, jakob.harden@student.tugraz.at, office@jakobharden.at
% Author website: https://jakobharden.at/wordpress/
% Author ORCID: https://orcid.org/0000-0002-5752-1785
%
% This file is part of the PhD thesis of Jakob Harden.
%#######################################################################################################################
%
% Beamer documentation: https://www.beamer.plus/Structuring-Presentation-The-Local-Structure.html
%
% preamble
\documentclass[11pt,aspectratio=169]{beamer}
\usepackage[utf8]{inputenc}
\usepackage[LGR,T1]{fontenc}
\usepackage[ngerman,english]{babel}
\usepackage{hyphenat}
\usepackage{lmodern}
\usepackage{blindtext}
\usepackage{multicol}
\usepackage{graphicx}
\usepackage{tikz}
\usetikzlibrary{calc,fpu,plotmarks}
\usepackage{pgfplots}
\pgfplotsset{compat=1.17}
\usepgflibrary{fpu}
\usepackage{amsmath}
\usepackage{algorithm}
\usepackage{algpseudocode}
\usepackage{hyperref}
\usepackage[backend=biber,style=numeric]{biblatex}
\addbibresource{biblio.bib}

%
% text blocks
\def\PresTitle{Local maxima detection for sinusoidal signals corrupted by noise}
\def\PresSubTitle{A numerical study on (damped) sinusoidal signals}
\def\PresDate{25${}^{\text{st}}$ of May, 2025}
\def\PresFootInfo{My PhD thesis, research in progress ...}
\def\PresAuthorFirstname{Jakob}
\def\PresAuthorLastname{Harden}
\def\PresAuthor{\PresAuthorFirstname{} \PresAuthorLastname{}}
\def\PresAuthorAffiliation{Graz University of Technology}
\def\PresAuthorAffiliationLocation{\PresAuthorAffiliation{} (Graz, Austria)}
\def\PresAuhtorWebsite{jakobharden.at}
\def\PresAuhtorWebsiteURL{https://jakobharden.at/wordpress/}
\def\PresAuhtorEmailFirst{jakob.harden@tugraz.at}
\def\PresAuhtorEmailSecond{jakob.harden@student.tugraz.at}
\def\PresAuhtorEmailThird{office@jakobharden.at}
\def\PresAuthorOrcid{0000-0002-5752-1785}
\def\PresAuthorOrcidURL{https://orcid.org/0000-0002-5752-1785}
\def\PresAuthorLinkedin{jakobharden}
\def\PresAuthorLinkedinURL{https://www.linkedin.com/in/jakobharden/}
\def\PresCopyrightType{ccby} % one of: copyright, ccby, ccysa
%
% Beamer theme adaptations
%   type:        Presentation
%   series:      Research in progress (RIP)
%   description: This theme is designed to present preliminary research results.
% !BIB TS-program = biblatex
% !TeX spellcheck = en_US
%
%#######################################################################################################################
% LICENSE
%
% "adaptthemePresRIP.tex" (C) 2024 by Jakob Harden (Graz University of Technology) is licensed under a Creative Commons Attribution 4.0 International license.
%
% License deed: https://creativecommons.org/licenses/by/4.0/
% Author email: jakob.harden@tugraz.at, jakob.harden@student.tugraz.at, office@jakobharden.at
% Author website: https://jakobharden.at/wordpress/
% Author ORCID: https://orcid.org/0000-0002-5752-1785
%
% This file is part of the PhD thesis of Jakob Harden.
%#######################################################################################################################
%
% Beamer theme adaptations
%   type:        Presentation
%   series:      Research in progress (RIP)
%   description: This theme is designed to present preliminary research results.
%
% Beamer documentation: https://www.beamer.plus/Structuring-Presentation-The-Local-Structure.html
%
%-----------------------------------------------------------------------------------------------------------------------
% color definitions
\definecolor{RIPbgcol}{RGB}{255, 233, 148} % LibreOffice, Light Gold 3
\definecolor{RIPsepcol}{RGB}{255, 191, 0} % LibreOffice, Gold
\definecolor{RIPtitlecol}{RGB}{120, 75, 4} % LibreOffice, Dark Gold 3
%
% geometry definition
\newlength{\RIPheadheight}
\setlength{\RIPheadheight}{14mm}
\newlength{\RIPfootheight}
\setlength{\RIPfootheight}{9mm}
%
%-----------------------------------------------------------------------------------------------------------------------
% commands
% two-column mode, left column
\newenvironment{RIPcolleft}{%
		\begin{column}{.65\textwidth}%
	}{%
		\end{column}%
	}
%
% two-column mode, right column
\newenvironment{RIPcolright}{%
		\hspace{.05\textwidth}%
		\begin{column}{.3\textwidth}%
	}{%
		\end{column}%
	}
%
% copyright information text block
\newcommand{\RIPcopyrightinfo}[1]{%
	Copyright \textcopyright{} \the\year{} \PresAuthor{} (\PresAuthorAffiliationLocation)\\
	%This document is licensed under a Creative Commons Attribution 4.0 International license.
	\ifstrequal{#1}{copyright}{All rights reserved.}{}
	\ifstrequal{#1}{ccby}{%
		This document is licensed under a Creative Commons Attribution 4.0 International license.\\
		See also: \href{https://creativecommons.org/licenses/by/4.0/deed}{CC BY 4.0, license deed}
	}{}%
	\ifstrequal{#1}{ccbysa}{%
		This document is licensed under a Creative Commons Attribution-Share Alike 4.0 International license.\\
		See also: \href{https://creativecommons.org/licenses/by-sa/4.0/deed}{CC BY-SA 4.0, license deed}
	}{}
	\\
	\vspace{1em}
	The above license applies to the entire content of this document. Deviations from this license are explicitly marked.
}
%
% author information text block
\newcommand{\RIPauthorinfo}[1]{%
	\renewcommand{\arraystretch}{1.25}
	\begin{tabular}{l l}
		First name & \PresAuthorFirstname{} \\
		Last name & \PresAuthorLastname{} \\
		Affiliation & \PresAuthorAffiliationLocation{} \\
		Website & \href{\PresAuhtorWebsiteURL{}}{\PresAuhtorWebsite{}} \\
		Email & \PresAuhtorEmailFirst{}, \PresAuhtorEmailSecond{}, \PresAuhtorEmailThird{} \\
		ORCID & \href{\PresAuthorOrcidURL}{\PresAuthorOrcid{}} \\
		LinkedIn & \href{\PresAuthorLinkedinURL}{\PresAuthorLinkedin{}}
	\end{tabular}
}
%
%-----------------------------------------------------------------------------------------------------------------------
% define and adapt theme
\usetheme{default} % presentation theme
\useoutertheme{sidebar} % outer theme
\useinnertheme{default} % inner theme
%
% size settings
\setbeamersize{%
	text margin left=5mm,
	text margin right=5mm,
	sidebar width left=0mm,
	sidebar width right=0mm}
%
% headline settings
\setbeamertemplate{headline}{%
	\begin{minipage}[t]{\textwidth}
		\begin{tikzpicture}
			\fill[RIPbgcol] (0,0) -- ++(16, 0) -- ++(0,-\RIPheadheight) -- ++(-16,0) -- cycle;
			\draw[RIPsepcol] (0,-\RIPheadheight) -- ++(16,0)
				node[pos=0.988,left,yshift=.6\RIPheadheight,RIPtitlecol] {\Large\insertpagenumber};
		\end{tikzpicture}
	\end{minipage}
}
%
% footline settings
\setbeamertemplate{footline}{%
	\begin{minipage}[t]{\textwidth}
		\begin{tikzpicture}
			\fill[RIPbgcol] (0,0) -- ++(16, 0) -- ++(0,-\RIPfootheight) -- ++(-16,0) -- cycle;
			\draw[RIPsepcol] (0,0) -- ++(16,0)
				node[pos=0.012,black,right,yshift=-4mm] {\small\PresFootInfo{}}
				node[pos=0.988,black,left,yshift=-5mm]{
					\parbox{35mm}{%
						\raggedleft
						\small\hfill\PresAuthor{}\newline
						\tiny\PresAuthorAffiliation{}
					}
				};
		\end{tikzpicture}
	\end{minipage}
}
%
% left sidebar settings
\setbeamertemplate{sidebar canvas left}{}
\setbeamertemplate{sidebar left}{}
%
% nagigation symbol settings
\setbeamertemplate{navigation symbols}{}
%
% abstract settings
\setbeamertemplate{abstract title}{\normalsize}
\setbeamertemplate{abstract begin}{\small}
\setbeamertemplate{abstract end}{}
%
% color settings
\setbeamercolor{titlelike}{fg=RIPtitlecol}
\setbeamercolor{bibliography entry author}{fg=RIPtitlecol}
\setbeamercolor{bibliography entry note}{fg=RIPtitlecol}
\setbeamercolor{bibliography item}{fg=black}
\setbeamercolor{caption}{fg=black}
\setbeamercolor{caption name}{fg=RIPtitlecol}
%
% itemization settings
\setbeamertemplate{itemize item}{\color{RIPtitlecol}$\blacktriangleright$}
\setbeamertemplate{itemize subitem}{\color{RIPtitlecol}$\blacksquare$}
%
% enumeration settings
\setbeamertemplate{enumerate item}{\color{RIPtitlecol}\bfseries\insertenumlabel}
%
% bibliography settings
\setbeamertemplate{bibliography item}{\insertbiblabel}

%
% Load Octave to TeX tool
% TeX commands to conveniently use serialized dataset content
%
%#######################################################################################################################
% LICENSE
%
% "oct2texdefs.tex" (C) 2024 by Jakob Harden (Graz University of Technology) is licensed under a Creative Commons Attribution 4.0 International license.
%
% License deed: https://creativecommons.org/licenses/by/4.0/
% Author email: jakob.harden@tugraz.at, jakob.harden@student.tugraz.at, office@jakobharden.at
% Author website: https://jakobharden.at/wordpress/
% Author ORCID: https://orcid.org/0000-0002-5752-1785
%
% This file is part of the PhD thesis of Jakob Harden.
%#######################################################################################################################
%
%
%-------------------------------------------------------------------------------
% Load etoolbox and other required pgf packages
\usepackage{etoolbox} % if clauses
\usepackage{pgf, pgfmath, pgfplots, pgfplotstable} % pgf functions
%
%-------------------------------------------------------------------------------
% Structure path prefix
% Note: The prefix is used to abbreviate long structure paths (variable names)
%
% Define default value of structure path prefix
% Do not change that value unless you know what you are doing!
\gdef\OTpfx{oct2tex}
%
% Set structure path prefix to a user defined value
%   Parameter #1: user defined prefix (string without whitespace)
%   Usage: \OTsetpfx{oct2tex.my.pre.fix}
\newcommand{\OTsetpfx}[1]{\ifstrempty{#1}{\gdef\OTpfx{oct2tex}}{\gdef\OTpfx{#1}}}
%
% Reset structure path prefix to default value
\newcommand{\OTresetpfx}{\gdef\OTpfx{oct2tex}}
%
%-------------------------------------------------------------------------------
% Use serialized content from data structures in the document
%
% Use structure variable
%   Parameter #1: variable name (structure path)
%   Usage: \OTuse{my.struct.path.to.content.value}
\newcommand{\OTuse}[1]{\csname \OTpfx.#1\endcsname}
%
% Use structure variable, fixed digit floating point number
%   Parameter #1: variable name (structure path)
%   Parameter #2: number of digits to display
%   Usage: \OTusefixed{my.struct.path.to.content.value}{2}
\newcommand{\OTusefixed}[2]{%
	\pgfkeys{%
		/pgf/number format/.cd,%
		fixed,%
		precision=#2,%
		1000 sep={.}%
	}%
	\pgfmathprintnumber{\OTuse{#1}}%
}
%
% Read tabulated value from structure and store result in the LaTeX command \OTtab
%   Parameter #1: variable name (structure path)
%   Usage: \OTread{my.struct.path.to.table}
\newcommand{\OTread}[1]{\pgfplotstableread[col sep=semicolon,trim cells]{\OTpfx.#1}\OTtab}
%
% Read CSV file and store result in the LaTeX command \OTtabcsv
%   Parameter #1: CSV file name
%   Usage: \OTreadcsv{csv_filename}
\newcommand{\OTreadcsv}[1]{\pgfplotstableread[col sep=semicolon,trim cells]{#1}\OTtabcsv}

%
% graphics path
\graphicspath{{../octave/results/test_peakholdmax}}
%
% computation result path
\newcommand{\RPATH}{../octave/results/test_peakholdmax}
%
%#######################################################################################################################
\begin{document}
	% set title page items
	\author{\PresAuthor{} (\PresAuthorAffiliation{})}
	\title{\PresTitle{}}
	\subtitle{\PresSubTitle{}}
	%\logo{}
	%\institute{}
	\date{\PresDate{}}
	%\subject{}
	%\setbeamercovered{transparent}
	%\setbeamertemplate{navigation symbols}{}
	%
	%-------------------------------------------------------------------------------------------------------------------
	\begin{frame}[plain]
		\maketitle
	\end{frame}
	%
	%-------------------------------------------------------------------------------------------------------------------
	\section*{Abstract}
	\begin{frame}
		\frametitle{Abstract}
		% german text
		\begin{abstract}
			%TODO: write abstract
		\end{abstract}
	\end{frame}
	%
	%-------------------------------------------------------------------------------------------------------------------
	\section{Introduction}
	\begin{frame}
		\frametitle{Introduction}
		\begin{itemize}
			\item \textcolor{RIPtitlecol}{WHAT}
			\begin{itemize}
				\item xxx
				\item xxx
			\end{itemize}
			\item \textcolor{RIPtitlecol}{WHY}
			\begin{itemize}
				\item xxx
				\item xxx
			\end{itemize}
			\item \textcolor{RIPtitlecol}{HOW}
			\begin{itemize}
				\item xxx
				\item xxx
			\end{itemize}
			\item \textcolor{RIPtitlecol}{USAGE}
				\begin{itemize}
					\item xxx
					\item xxx
				\end{itemize}
			\item \textcolor{RIPtitlecol}{HIGHLIGHT} \textbf{xxx}
		\end{itemize}
	\end{frame}
	%
	%-------------------------------------------------------------------------------------------------------------------
	\section{Materials \& Methods}
	\begin{frame}
		\frametitle{Materials \& Methods I}
		\begin{columns}
			\begin{RIPcolleft}
				\begin{itemize}
					\setlength\itemsep{0.5em}
					\item \textcolor{RIPtitlecol}{Materials}
					\begin{itemize}
						\setlength\itemsep{0.5em}
						\item xxx
					\end{itemize}
				\end{itemize}
			\end{RIPcolleft}
			\begin{RIPcolright}
				\textbf{xxx}\\
				\vspace{1em}
				%\includegraphics[width=40mm,trim={15mm 10mm 20mm 20mm}, clip]{xxx}
			\end{RIPcolright}
		\end{columns}
	\end{frame}
	%
	\begin{frame}
		\frametitle{Materials \& Methods II}
		\begin{itemize}
			\setlength\itemsep{0.5em}
			\item \textcolor{RIPtitlecol}{Methods}
			\begin{itemize}
				\setlength\itemsep{0.5em}
				\item xxx
			\end{itemize}
		\end{itemize}
		\vspace*{.5em}
		\small \textbf{Note:} xxx.
	\end{frame}
	%
	\begin{frame}
		\frametitle{Materials \& Methods III}
		To study the algorithm, a parameter variation was carried out using the following parameters.
		\begin{itemize}
			\setlength\itemsep{0.5em}
			\item \textcolor{RIPtitlecol}{Numerical study - parameter variation}
			\begin{itemize}
				\setlength\itemsep{0.5em}
				\item Constant: $A = 1 \quad [V]$
				\item Constant: unbiased ACF estimator
				\item Variation 1: $N_1 = (64, 512, 1024, 4096)$ samples
				\item Variation 2: $SNR = (5, 10, 15, 20)$ dB
				\item Variation 3: $DF = (0, 2, 4)$
				\item Variation 4: $N_{cy} = (1,\ldots,5)$ samples, subdivided into 81 steps
				\item Variation 5: $N_{mc} = 500$ turns, Monte-Carlo test
			\end{itemize}
		\end{itemize}
	\end{frame}
	%
	%-------------------------------------------------------------------------------------------------------------------
	\section{Results}
	%
	\begin{frame}
		\frametitle{Results \textendash{} Signal power estimates, example I}
		\begin{columns}[t]
			\begin{RIPcolleft}
				\begin{figure}
					%\includegraphics[width=100mm,trim= 5mm 0mm 5mm 50mm]{xxx}
				\end{figure}
			\end{RIPcolleft}
			\begin{RIPcolright}
				\textbf{xxx}\\
				xxx\\
				\vspace*{.5em}
				\textbf{Observations}\\
				\begin{itemize}
					\item xxx
				\end{itemize}
			\end{RIPcolright}
		\end{columns}
	\end{frame}
	%
	%-------------------------------------------------------------------------------------------------------------------
	\section{Conclusions}
	\begin{frame}
		\frametitle{Conclusions}
		% briefly summarize all observations
		xxx
		\vspace*{1em}
		\begin{itemize}
			\item xxx
		\end{itemize}
	\end{frame}
	%
	%-------------------------------------------------------------------------------------------------------------------
	\section{Outlook}
	\begin{frame}
		\frametitle{Outlook}
		% briefly describe further and connected research
		xxx
		\vspace*{1em}
		\begin{itemize}
			\item xxx
		\end{itemize}
		\vspace*{.5em}
		\small \textbf{Note:} To support the future development of the proposed method, the supplementary GNU Octave code\autocite{progcode} and the \LaTeX{} code\autocite{texcode} of this presentation is made available publicly under open licenses.
	\end{frame}
	%
	%-------------------------------------------------------------------------------------------------------------------
	\section*{References}
	\begin{frame}[noframenumbering]
		\frametitle{References}
		\printbibliography
	\end{frame}
	%
	%===================================================================================================================
	\appendix
	\section{\appendixname}
	%
	\begin{frame}
		\frametitle{\appendixname{} \textendash{} xxx}
		xxx
	\end{frame}
	%
	\begin{frame}[noframenumbering]
		\frametitle{\appendixname{} \textendash{} Author information}
		\RIPauthorinfo{}
	\end{frame}
	\begin{frame}[noframenumbering]
		\frametitle{\appendixname{} \textendash{} Document license}
		\expandafter\RIPcopyrightinfo\expandafter{\PresCopyrightType}
	\end{frame}
\end{document}
